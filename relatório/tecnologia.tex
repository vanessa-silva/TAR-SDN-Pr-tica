\section*{Tecnologia utilizada}

Para a realização deste trabalho prático, utilizou-se o emulador Mininet (\ref{mininet}), o protocolo OpenFlow (\ref{openflow}), o controlador Floodlight. %e o programa Wireshark (\ref{wireshark}).

\subsection{Mininet} \label{mininet}

Mininet é um emulador de redes, desenvolvido por pesquisadores da Universidade de Stanford nos Estados Unidos, com o objetivo de apoiar pesquisas colaborativas, permitindo protótipos autônomos de redes definidas por software (SDNs), de modo a que qualquer pessoa possa fazer o seu download, executar, avaliar, explorar e ajustar.
É um emulador do tipo CBE (Container-Based Emulation), onde um ambiente de \textit{hosts} virtuais, \textit{switches} e \textit{links} é executado em um servidor multicore moderno, usando código e kernel com elementos de rede emulados por software, é então uma forma mais leve de virtualização onde muitos recursos do sistema são compartilhados. 

O Mininet é capaz de emular \textit{links}, \textit{hosts}, \textit{switches} e controladores, utilizando processos que rodem em espaços de nomes da rede (\textit{network namespaces}) e pares Ethernet virtuais.

\begin{itemize}

	\item \textit{Links}: Um par Ethernet virtual atua como um cabo que conecta duas interfaces virtuais. Os pacotes enviados através de uma interface são entregues na outra e cada interface se comporta como uma interface Ethernet completa.
	
	\item \textit{Host}: É um processo da shell movido para o seu próprio espaço de nome da rede, ou seja, cada \textit{host} apresenta uma instância da interface de rede independente. Cada dispositivo apresenta uma ou mais interfaces virtuais.
	
	\item \textir{Switches}: \textit{Switches} OpenFlow são remotamente configurados e gerenciados pelo controlador, estes, também, podem ser configurados para agirem como \textit{switches} convencionais.
	
	\item Controladores: Controladores OpenFlow podem estar em qualquer parte da rede, ou seja, na rede física ou na rede virtual.

\end{itemize}

De forma a controlar e gerenciar todos os dispositivos emulados, o Mininet fornece uma CLI (\textit{Command Line Interface}) que conhece toda a rede, ou seja, através de uma única linha de comandos, pode-se controlar todos os dispositivos emulados. Outra vantagem é a possibilidade de criar topologias de rede através de uma API (\textit{Application Programming Interface}) para programação em Python.

\subsection{OpenFlow} \label{openflow}

OpenFlow é um protocolo que estabelece comunicações \textit{switches}-controlador.
Numa rede Openflow quando um dispositivo recebe um pacote, este não toma a decisão de forma autónoma, ele envia o pacote para o controlador que por sua vez utiliza certos critérios de forma a tomar de decisão, este cria uma regra de encaminhamento que passa para o dispositivo solicitante.

O controlador dispõe de todas as informações necessárias para determinar as funções desempenhadas pelo \textit{switch} como, a porta para a qual o fluxo deve ser encaminhado ou descartar os pacotes.
Ou seja, serve basicamente para adicionar ou remover entradas da tabela de fluxos de todos os \textit{switches} que se encontram a ser controlados por ele.
Existem muitos tipos de controladores diferentes onde, o estilo de desenvolvimento e as funcionalidades que eles oferecem, são maioritariamente determinados pela linguagem utilizada.

A tabela de fluxos é uma base de dados de entradas de fluxos com alguns componentes principais, onde podemos destacar tarefas como: encaminhamento de um pacote para uma determinada porta do \textit{switch}, descartar um pacote, criptografar esse pacote, etc.

Um \textit{switch} que utiliza Openflow separa o encaminhamento do roteamento de forma a melhorar a gestão da rede.
A função de encaminhamento permanece no \textit{switch}, mas as decisões de roteamento passam a ser da responsabilidade do controlador.

Openflow permite a manipulação de todo o fluxo de dados, permitindo escolher as rotas a serem seguidas pelos pacotes e o processamento a que eles devem ser submetidos.

%\subsection{Wireshark} \label{wireshark}


\section*{Objetivo do trabalho laboratorial}


\section*{Configurações/\textit{scripts}}


\section*{Resultados}


\section*{Conclusão}